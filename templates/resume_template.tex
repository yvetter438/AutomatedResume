\documentclass[10pt, letterpaper]{article}

% Packages:
\usepackage[
    ignoreheadfoot, % set margins without considering header and footer
    top=2 cm, % seperation between body and page edge from the top
    bottom=2 cm, % seperation between body and page edge from the bottom
    left=2 cm, % seperation between body and page edge from the left
    right=2 cm, % seperation between body and page edge from the right
    footskip=1.0 cm, % seperation between body and footer
    % showframe % for debugging 
]{geometry} % for adjusting page geometry
\usepackage{titlesec} % for customizing section titles
\usepackage{tabularx} % for making tables with fixed width columns
\usepackage{array} % tabularx requires this
\usepackage[dvipsnames]{xcolor} % for coloring text
\definecolor{primaryColor}{RGB}{0, 0, 0} % define primary color
\usepackage{enumitem} % for customizing lists
\usepackage{fontawesome5} % for using icons
\usepackage{amsmath} % for math
\usepackage[
    pdftitle={Yannick Vetter's Resume},
    pdfauthor={Yannick Vetter},
    pdfcreator={LaTeX},
    colorlinks=true,
    urlcolor=primaryColor
]{hyperref} % for links, metadata and bookmarks
\usepackage[pscoord]{eso-pic} % for floating text on the page
\usepackage{calc} % for calculating lengths
\usepackage{bookmark} % for bookmarks
\usepackage{lastpage} % for getting the total number of pages
\usepackage{changepage} % for one column entries (adjustwidth environment)
\usepackage{paracol} % for two and three column entries
\usepackage{ifthen} % for conditional statements
\usepackage{needspace} % for avoiding page brake right after the section title
\usepackage{iftex} % check if engine is pdflatex, xetex or luatex

% Ensure that generate pdf is machine readable/ATS parsable:
\ifPDFTeX
    \input{glyphtounicode}
    \pdfgentounicode=1
    \usepackage[T1]{fontenc}
    \usepackage[utf8]{inputenc}
    \usepackage{lmodern}
\fi

\usepackage{charter}

% Some settings:
\raggedright
\AtBeginEnvironment{adjustwidth}{\partopsep0pt} % remove space before adjustwidth environment
\pagestyle{empty} % no header or footer
\setcounter{secnumdepth}{0} % no section numbering
\setlength{\parindent}{0pt} % no indentation
\setlength{\topskip}{0pt} % no top skip
\setlength{\columnsep}{0.15cm} % set column seperation
\pagenumbering{gobble} % no page numbering

\titleformat{\section}{\needspace{4\baselineskip}\bfseries\large}{}{0pt}{}[\vspace{1pt}\titlerule]

\titlespacing{\section}{
    % left space:
    -1pt
}{
    % top space:
    0.3 cm
}{
    % bottom space:
    0.2 cm
} % section title spacing

\renewcommand\labelitemi{$\vcenter{\hbox{\small$\bullet$}}$} % custom bullet points
\newenvironment{highlights}{
    \begin{itemize}[
        topsep=0.10 cm,
        parsep=0.10 cm,
        partopsep=0pt,
        itemsep=0pt,
        leftmargin=0 cm + 20pt
    ]
}{
    \end{itemize}
} % new environment for highlights


\newenvironment{highlightsforbulletentries}{
    \begin{itemize}[
        topsep=0.10 cm,
        parsep=0.10 cm,
        partopsep=0pt,
        itemsep=0pt,
        leftmargin=10pt
    ]
}{
    \end{itemize}
} % new environment for highlights for bullet entries

\newenvironment{onecolentry}{
    \begin{adjustwidth}{
        0 cm + 0.00001 cm
    }{
        0 cm + 0.00001 cm
    }
}{
    \end{adjustwidth}
} % new environment for one column entries

\newenvironment{twocolentry}[2][]{
    \onecolentry
    \def\secondColumn{#2}
    \setcolumnwidth{\fill, 4.5 cm}
    \begin{paracol}{2}
}{
    \switchcolumn \raggedleft \secondColumn
    \end{paracol}
    \endonecolentry
} % new environment for two column entries

\newenvironment{threecolentry}[3][]{
    \onecolentry
    \def\thirdColumn{#3}
    \setcolumnwidth{, \fill, 4.5 cm}
    \begin{paracol}{3}
    {\raggedright #2} \switchcolumn
}{
    \switchcolumn \raggedleft \thirdColumn
    \end{paracol}
    \endonecolentry
} % new environment for three column entries

\newenvironment{header}{
    \setlength{\topsep}{0pt}\par\kern\topsep\centering\linespread{1.5}
}{
    \par\kern\topsep
} % new environment for the header

\newcommand{\placelastupdatedtext}{% \placetextbox{<horizontal pos>}{<vertical pos>}{<stuff>}
  \AddToShipoutPictureFG*{% Add <stuff> to current page foreground
    \put(
        \LenToUnit{\paperwidth-2 cm-0 cm+0.05cm},
        \LenToUnit{\paperheight-1.0 cm}
    ){\vtop{{\null}\makebox[0pt][c]{
        \small\color{gray}\textit{Last updated in February 2025}\hspace{\widthof{Last updated in February 2025}}
    }}}%
  }%
}%

% save the original href command in a new command:
\let\hrefWithoutArrow\href

% new command for external links:


\begin{document}
    \newcommand{\AND}{\unskip
        \cleaders\copy\ANDbox\hskip\wd\ANDbox
        \ignorespaces
    }
    \newsavebox\ANDbox
    \sbox\ANDbox{$|$}

    \begin{header}
        \fontsize{25 pt}{25 pt}\selectfont Yannick Vetter

        \vspace{5 pt}

        \normalsize
        \mbox{\hrefWithoutArrow{mailto:yannick.vetter@ndsu.edu}{yannick.vetter@ndsu.edu}}%
        \kern 5.0 pt%
        \AND%
        \kern 5.0 pt%
        \mbox{\hrefWithoutArrow{tel:+1701-730-4709}{701-730-4709}}%
        \kern 5.0 pt%
        \AND%
        \kern 5.0 pt%
        \mbox{\hrefWithoutArrow{https://yannickvetter.com/}{yannickvetter.com}}%
        \kern 5.0 pt%
        \AND%
        \kern 5.0 pt%
        \mbox{\hrefWithoutArrow{https://linkedin.com/in/yannickvetter}{linkedin.com/in/yannickvetter}}%
        \kern 5.0 pt%
        \AND%
        \kern 5.0 pt%
        \mbox{\hrefWithoutArrow{https://github.com/yvetter438}{github.com/yvetter438}}%
    \end{header}

    \vspace{5 pt - 0.3 cm}

    \section{Education}

    \begin{twocolentry}{
        Aug 2021 – May 2025
    }
        \textbf{North Dakota State University}, BS in Computer Science
    \end{twocolentry}

    \vspace{0.10 cm}
    \begin{onecolentry}
        \begin{highlights}
            \item \textbf{Coursework:} Computer Architecture, Data Structures and Algorithms, Networking and Parallel Computation
        \end{highlights}
    \end{onecolentry}

    \section{Experience}

    \begin{twocolentry}{
        \VAR{job1.dates}
    }
        \textbf{\VAR{job1.title}}, \VAR{job1.company} -- \VAR{job1.location}
    \end{twocolentry}

    \vspace{0.10 cm}
    \begin{onecolentry}
        \begin{highlights}
            \BLOCK{ for point in job1.points }
            \item \VAR{point}
            \BLOCK{ endfor }
        \end{highlights}
    \end{onecolentry}

    \vspace{0.2 cm}

    \begin{twocolentry}{
        \VAR{job2.dates}
    }
        \textbf{\VAR{job2.title}}, \VAR{job2.company} -- \VAR{job2.location}
    \end{twocolentry}

    \vspace{0.10 cm}
    \begin{onecolentry}
        \begin{highlights}
            \BLOCK{ for point in job2.points }
            \item \VAR{point}
            \BLOCK{ endfor }
        \end{highlights}
    \end{onecolentry}

    \vspace{0.2 cm}

    \begin{twocolentry}{
        \VAR{job3.dates}
    }
        \textbf{\VAR{job3.title}}, \VAR{job3.company} -- \VAR{job3.location}
    \end{twocolentry}

    \vspace{0.10 cm}
    \begin{onecolentry}
        \begin{highlights}
            \BLOCK{ for point in job3.points }
            \item \VAR{point}
            \BLOCK{ endfor }
        \end{highlights}
    \end{onecolentry}

    \vspace{0.2 cm}

    \begin{twocolentry}{
        \VAR{job4.dates}
    }
        \textbf{\VAR{job4.title}}, \VAR{job4.company} -- \VAR{job4.location}
    \end{twocolentry}

    \vspace{0.10 cm}
    \begin{onecolentry}
        \begin{highlights}
            \BLOCK{ for point in job4.points }
            \item \VAR{point}
            \BLOCK{ endfor }
        \end{highlights}
    \end{onecolentry}

    \section{Projects}

    \begin{twocolentry}{
        \href{https://github.com/yvetter438/Crave}{github.com/yvetter438/Crave}
    }
        \textbf{Crave - Social Media Recipe Video App}
    \end{twocolentry}

    \vspace{0.10 cm}
    \begin{onecolentry}
        \begin{highlights}
            \item Built and launched a mobile app for sharing and discovering recipes via short-form videos.
            \item Tools Used: React Native, Typescript, Supabase, Expo
        \end{highlights}
    \end{onecolentry}

    \vspace{0.2 cm}

    \begin{twocolentry}{
        \href{https://github.com/yvetter438/AutomatedResume}{github.com/yvetter438/AutomatedResume}
    }
        \textbf{Automated Resume Tailoring Tool}
    \end{twocolentry}

    \vspace{0.10 cm}
    \begin{onecolentry}
        \begin{highlights}
            \item Developed a full-stack resume management application using Python/Flask and SQLite, featuring drag-and-drop reordering, AI-powered resume optimization, and PDF generation. Feautures a job application tracking system.
            \item Tools Used: Python, Flask, SQLite, LaTeX, Jinja, LLM's
        \end{highlights}
    \end{onecolentry}

    \vspace{0.2 cm}

    \begin{twocolentry}{
        ndsuskydivingclub.com
    }
        \textbf{NDSU Skydiving Club Website}
    \end{twocolentry}

    \vspace{0.10 cm}
    \begin{onecolentry}
        \begin{highlights}
            \item Designed and developed website using HTML/CSS and deployed through Github Pages the official NDSU Skydiving Club website.
        \end{highlights}
    \end{onecolentry}

    \section{Technologies}

    \begin{onecolentry}
        \textbf{Languages:} TypeSript, Python, Java, HTML, CSS, SQL, JavaScript
    \end{onecolentry}

    \vspace{0.2 cm}

    \begin{onecolentry}
        \textbf{Technologies:} Expo, Supabase, XCode, GitHub
    \end{onecolentry}

\end{document}